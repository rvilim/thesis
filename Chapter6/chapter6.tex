%!TEX root = ../thesis.tex

\chapter{Conclusions and Future Work}
\label{chap:conclusion}
In this work we have discussed how novel material properties can affect magnetic field generation in planets. We have focussed on the magnetic fields of the planet Mercury, as well as the magnetic fields of terrestrial planets in other planetary systems. This work has provided predictions about the likelihood of detecting exoplanetary magnetic fields (chapter \ref{chap:superearth}). We have also discussed a new model to explain the weak dipolar field measured at Mercury (chapter \ref{chap:doublesnowstates}). Finally, we tested a second model which was proposed \citep{smith2012, hauck2013} to explain Mercury's weak magnetic field (chapter \ref{chap:floatationlayers}) and found that the magnetic fields it produced did not match those measured at Mercury.

\subsection{The Magnetic Fields of Exoplanets with Metallized Mantles}
In chapter \ref{chap:superearth} we used recent evidence which indicates that the materials which commonly comprise the mantles of terrestrial planets can metallise and conduct electricity under mantle conditions \citep{nellis2010, tsuchiya2011, ohta2012}. We found that magnetic field generated by the core dynamo would become simultaneously trapped in the solid, immobile mantle and the strong, axisymmetric zonal flows of the core. This leads to a strong axisymmetric toroidal field within the core by magnetic stretching at the core-mantle boundary. This toroidal field can only be converted to a non-axisymmetric poloidal field, due to geometrical constraints which become apparent by applying the Bullard-Gelman formalism to this scenario (Appendix \ref{chap:appendix1}). This non-axisymmetric poloidal field appears as a time varying flow when it is advected by the zonal flows in the core, and is attenuated by the screening effect of the electrically conducting mantle. We found that this means an exoplanet with a solid, electrically conducting mantle may have a weak observed magnetic field despite having a much stronger internal magnetic field. The results from this study are published in \citet{vilim2013}.
 
\subsection{The Magnetic Field of Mercury}
In this work we discussed two possible ways of generating Mercury's weak, equatorially asymmetric magnetic field. The first appealed to a ``double snow state'', a unique core solidification regime which Mercury may occupy due to a combination of its core-composition and the pressures found within it. In this scenario iron freezes first part way through the liquid outer core and injects buoyancy there. We modelled this as a stably stratified layer and found that this split the core into two dynamo generation regions which generated magnetic field of opposite polarity. These magnetic fields superposed to create a weak, dipolar magnetic field. We also found that the magnetic field generated by these models did not match the recently reported northward displacement of Mercury's magnetic equator \citep{smith2012}. The results from this study are published in \citet{vilim2010}

In this work we also tested a recently proposed model of Mercury's interior. This model proposed a layer of FeS at the top of Mercury's core which would preferentially attenuate the non-dipolar components of the generated magnetic field. It was proposed that this could explain the moment of inertia of Mercury's outer shell and its weak observed magnetic field. We found that the addition of a solid, electrically conducting layer at the top of the core caused significant Lorentz stretching there, which lead to a strong, dipolar magnetic field and a decrease in the characteristic timescale of the dynamo. This means that a solid electrically conducting layer which is thin enough to be consistent with the moment of inertia observations is insufficiently thin to screen the magnetic field effectively, leading to a strong, dipolar observed magnetic field.


\subsection{Future Work}
There are several avenues which should be explored as extensions to the work presented in this thesis.

\subsubsection{Metallized Mantles at High Local Rossby Numbers}
In chapter \ref{chap:superearth} we only focussed on dynamo models with $Ro_l<0.12$, which means that these dynamos would be predominantly dipolar in the absence of a metallised mantle layer. While there are almost certainly exoplanets with $Ro_l>0.12$, an analysis of these models was outside the scope of the work in chapter \ref{chap:superearth}. If $Ro_l>0.12$ the scaling laws of \citet{christensen06scaling} would predict that these dynamos should generate multipolar magnetic fields. Since mode $l$ will decay in field strength from the core as $\left(a/r\right)^{l+2}$ the strongest components of these dynamos would decay in strength very rapidly with distance and should be much more difficult to detect. 

When testing the feasibility of a solid, electrically conducting layer at the bottom of Mercury's mantle (chapter \ref{chap:floatationlayers}) we discovered that at high local Rossby numbers an electrically conducting mantle changed the resultant magnetic field from being small scale and highly time variable to large scale and predominantly dipolar. As we discussed in chapter \ref{chap:floatationlayers} this was because the electrically conducting mantle provides a source of shear, which promotes the generation of large scale, dipolar toroidal field. This toroidal field is then converted into large scale dipolar poloidal field leading to a strong dipolar magnetic field. 

This means that contrary to the conclusions of chapter \ref{chap:superearth}, a metallized mantle layer may make some dynamos easier to detect rather than more difficult as we concluded. In order to determine whether this assessment is true, we plan to re-analyse our current results in the context of exoplanets.

\subsubsection{Matching $g_1^1$ in Double Snow State Models}
All of the results presented in chapter \ref{chap:doublesnowstates} were finished before the publication of the $g_2^0$ obtained by the MESSENGER spacecraft \citep{anderson2012}. While these results currently do not match the $g_1^1$ component, it is not known whether a double snow state model with different parameters or in a different geometry could match the $g_1^1$ component. In future work we will run numerical dynamo models to attempt to match a model with a double snow state to the newest MESSENGER observations.