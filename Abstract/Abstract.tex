%!TEX root = ../thesis.tex
In this thesis I use a three dimensional numerical dynamo model to explore the effect of novel material properties and core states on magnetic field generation in the planet Mercury, and in rocky extra-solar planets.

In the first part of this work I focus on the recent evidence of pressure induced metallisation in materials which commonly comprise planetary mantles \citep{nellis2010, tsuchiya2011, ohta2012}. In this scenario the materials which make up the lower mantle of a planet conduct electricity with a conductivity similar to that of iron. I show that a metallised mantle changes the way in which magnetic field is generated by providing a new source of magnetic shear between the fluid outer core and the solid mantle. I then show that this has the effect of making planetary magnetic fields more difficult to observe from Earth. 

The second and third parts of this work focus on the planet Mercury. First, I incorporate recent evidence \citep{chenetal2008} of buoyancy sources mid-way through Mercury's liquid core (known as ``snow zones'') to show that they can explain the weak observed magnetic field of Mercury. In a second project on Mercury I test whether recent evidence of a dense solid layer at the top of  Mercury's core, attributed to a solid, electrically conducting layer of FeS \citep{smith2012, hauck2013}, could help explain Mercury's weak magnetic field. I find that the addition of this layer causes the dynamo to generate a strong, dipolar magnetic field, which does not match the observations made by the MESSENGER spacecraft \citep{anderson2012}.